
\subsection{Other Considerations}
There are other considerations that received less focus so far 
and were often overlooked in previous systems but needs to be 
worked on for adoption in the wild. As work on performance tuning 
saturates, we expect there to be more focus on these topics in future. 

\subsubsection{Fault Tolerance \& Reliability}
By spreading memory across multiple servers/nodes, memory 
disaggregation expands the fault domain of an application 
and makes it more prone to failures. Our system should account
for this; we only generally need to worry about remote node 
failures as local node failures occur in traditional 
setup too and the goal is for our system to be no less 
fault-tolerant than traditional setup.

One option is in-memory replication~\cite{leap,kona}.
This, however, would consume at least twice as much memory,
wasting precious DRAM and may prove too high of a cost. 
LegoOS~\cite{legoos} only maintains the log on secondary 
replica to minimize the memory overhead risking higher 
recovery time after a failure. Another alternative 
is to write the remote pages/data to persistent storage (disk) 
in the background along with storing it in remote 
memory~\cite{infiniswap}. However, remote write load 
may be too high for disk to handle which may result in 
disk queue build-ups. Recovering from failure would also be 
slower but this may be tolerable if the failures are rare.
In both cases though, replication/writing to disk only happen 
during evictions which are generally off the critical path,
and hence won't directly affect application performance.
It is not clear, however, whether we can get away without 
any of these and just fail the applications using a memory
node when it crashes.
% \anil{LegoOS did some analysis on 
% mean time to failure for hardware that may help here.}

Since network is involved now, transient network failures
or congestion may affect remote accesses. Normally this only degrades 
performance due to timeouts/retries during page faults/misses
but does not affect fault tolerance. In Kona~\cite{kona} however,
remote accesses are served using special hardware through 
cache-coherent protocol that is sensitive to memory access 
latencies and may end up crashing the application.

\subsubsection{Network Efficiency}
Network bandwidth may become a limiting factor and so better 
network efficiency would keep the network less congested and 
latencies low. Amount of networked data is primarily affected 
by quality of caching (cache misses bring in the data) and 
the evictions (writing the dirty data back). Bigger cache
blocks, like the page size in most systems, will cause 
data amplification (by bringing in redundant data) and hurt 
network efficiency, especially during the write-backs where 
dirty data is usually only a small fraction of the block size. 
Paging approaches cannot track dirty data on finer granularity 
but Kona~\cite{kona} (with special hardware) and AIFM~\cite{aifm}
with userspace runtimes can, and avoid I/O amplification by just 
writing this data. For example, Kona uses a log to track multiple 
dirty cache lines and sends it to the memory node in batches.

\subsubsection{Security \& Isolation}
Kernel is (has to be) trusted, and as long as the 
runtime (and agents on all the components) work in 
kernel space, remote memory can be indirectly provided 
to application where these indirection mapping/translation 
is controlled by the kernel, providing a security 
isolation similar to that provided by the traditional 
virtual memory. All kernel-based systems provide this 
support to enable safe access to remote memory in presence
of multiple applications. User space runtimes cannot 
implement isolation and support sharing between 
applications as they are restricted to a single 
application. Such runtimes may need to 
bank on kernel-based, lower-level interfaces like 
Remote Regions~\cite{remregions} or LITE~\cite{literdma}
for proper isolation in presence of multiple applications 
using such runtimes. Even if an attacker does not 
control these components, disaggregated memory increases the 
attack surface and may enable side-channel attacks such as  
Pythia~\cite{Pythia}, so more work on security is certainly 
needed. 

% \vspace{3pt}
% \noindent \uline{Performance Isolation.}
% caching policies only worry about evicting bad hits 
% and not separating app-specific. why should they? do 
% current OSes do it?
% only Leap separates page caches for applications
% Or they'll depend on the isolation of underlying 
% interconnects.
% But is perf isolation really provided in 
% traditional virtual memory?

% \subsubsection{Cluster-level Throughput}
% Systems only focus on memory performance from the 
% POV of single application. Would it be different for 
% overall cluster throughput? Fastswap?
% Given a set of jobs with cpu and memory 
% requirements, how do you put them on the 
% compute nodes with with max resource 
% utilization? 

% However, none ofthese approaches specify how to 
% schedule jobs when their memory can be split across 
% local and shared remote memory;

