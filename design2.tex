
\subsection{Application Performance}
\label{sec:performance}
When running on a disaggregated system, we ideally expect 
no or minimal degradation in application performance 
when compared to local memory performance. Depending 
on the application, one can look at its job completion time, 
throughput or tail latencies as a proxy for performance. 
(Although, metrics like total cost of ownership (TCO) are 
more holistic and account for things like memory 
utilization across the cluster, special hardware costs, etc.
but they are harder to measure?). In this section, we 
look at various factors that affect the memory performance,
through these metrics.


\subsubsection{Caching}
Remote accesses across the network today are still on 
the order of 10-100x slower with depending on the 
networking stack and interconnect, so application 
performance would be terrible if all accesses were 
to be remote~\cite{netdisagg}. Fortunately, most 
applications exhibit spatial and temporal locality 
in accesses which can be exploited by caching 
remote data, and as such the quality of caching can 
greatly affect the performance. Depending on the 
interface exposed, the runtime may choose to do 
caching or leave it to the application itself. 
Custom interfaces can enable application-specific 
caching hints. For example, AIFM~\cite{aifm} allows 
applications to exclude caching specified data and 
avoid cache pollution. 

\vspace{3pt}
\noindent \uline{Cache Block Size.} Fetching remote data in 
bigger chunks will help exploit spatial locality 
however it also runs the risk of bringing in redundant data 
and polluting the cache, so it needs to be properly 
balanced for the best cache hit ratio. Traditional
virtual memory-based approaches cannot go lower than 
the (4KB) page sized blocks as they hook into kernel 
paging; While LegoOS~\cite{legoos} and Kona~\cite{kona} 
escaped this fate through hardware modifications, 
other systems like AIFM~\cite{aifm} moved to user 
space implementations. Kona evaluates the effect of 
cache block size on the performance of Redis database 
and determines 1KB to be optimal, which is conveniently 
closer to the page size. However, Kona does not do any 
advanced prefetching (like Leap~\cite{leap}) which, in 
combination with smaller block sizes, may perform better 
than exploiting crude spatial locality with bigger blocks.
Block size also effects eviction policies like LRU (which
most systems use) as smaller blocks mean a larger number 
of blocks that need to be monitored for finding eviction
candidates. 

\vspace{3pt}
\noindent \uline{Prefetching.}
Prefetching remote data proactively can bring in 
correct pages into the cache and avoid cache misses 
in the critical path. Runtimes can use a transparent 
prefetcher that identifies access patterns and predict 
future accesses and/or they can provide prefetch calls 
in the interface that applications can inject in their 
code. While having a prefetcher is a more general 
solution, it has to balance between the accuracy of 
predictions and the (compute and memory) resources
it consumes (lower prefetching accuracy results in 
cache pollution and waste of cache and I/O bandwidth).
Leap~\cite{leap} is an advanced prefetcher for remote 
paging that monitors page faults and uses the faulted 
addresses to predict future pages. Even with coarse 
information like page faults, Leap was able to achieve 
1.5-2x improvement for different applications. 
Systems like AIFM~\cite{aifm} and Remote 
Regions~\cite{remregions} expose prefetch API 
providing applications the choice of implementing custom
prefetching such as the data structure-specific ones in AIFM. 

\vspace{3pt}
\noindent \uline{Cache Size.}
The bigger the size of the cache, the better; but the 
amount of local DRAM is limited (this limit is 
especially strict in hardware-disaggregated architecture 
where the amount of local DRAM is small and fixed). 
Even with the best prefetching and eviction policies, 
the cache size has to be enough to cover a minimum portion 
of the working set to avoid performance degradation and, 
in extreme cases, thrashing. A common chart we see in the 
evaluation of previous systems is to show the slowdown of 
an application against the local memory (or, cache size) as 
\% of either app's peak memory usage (which varies across 
apps) or an arbitrary local memory capacity, and compare 
it to other systems; the point being no one wants to pick 
a particular cache size but leave that option to the reader 
to trade it off with performance. Looking at variety of 
applications across papers, it seems like the performance 
degradation conforms to a hockey stick pattern, remaining 
graceful until some 25-50\% of the working 
set~\cite{netdisagg,kona,legoos}. None of these systems give 
an idea as to the one-size-fits-all limit for the absolute 
size of local DRAM though. \anil{double check these 
assertions. cache size is important because it gives a 
sense of how many/what kind of app mixture can be run on 
compute node without thrashing} 

\subsubsection{Interface Overhead}
The interface itself can add some software overhead 
to the each memory access. Unlike virtual memory 
interfaces that use native load/store, user space 
interfaces involve either library calls or another 
level of pointer indirection (e.g., Java) that may add 
few cycles for each operation. For example, AIFM uses
C++ smart pointers and when compared to Fastswap that 
allows native pointers, it adds a marginal overhead 
that becomes evident when effects of their runtime and 
interconnect are minimized. (Fig 7~\cite{aifm}). 
Similarly, Kona~\cite{kona} that routes remote accesses
through the cache-coherent hardware that exposes remote 
memory as another physical DRAM whose accesses are 
slower compared to regular DRAM due to limited interconnect 
bandwidth \anil{not sure how much}.

Kernel-based custom interfaces like LITE~\cite{literdma} 
introduce syscall in the access path that adds 
significant overhead. LITE however is a low-level 
interface that leaves caching to the application and 
such accesses must be made in the cache miss path. 


\subsubsection{Remote Access Latency} 
Optimizing the actual latency in fetching remote data 
(i.e., the cache miss path) is important not only because 
caching cannot soften the performance impact if the miss 
latency is too high but also because it impacts tail-latencies 
that modern datacenter applications are sensitive to. 
This latency depends on both the implementation 
overheads on the client and server-side and the choice of 
the interconnect itself. We discuss these factors below, 
in addition to the previous optimizations to reduce or 
hide this latency.

\vspace{3pt}
\noindent \uline{Network Interconnect.}
Gao et al.~\cite{netdisagg} analyzed the effect of increasing
remote access latency for various applications and arrive at 
4-5 $\mu$s upper bound to maintain performance. (It was, 
however, done on paging-based systems with default page 
replacement algorithms not optimized for remote memory, and 
hence should be taken as a rough estimate).
Such low latencies are now possible in tight-knit
local area networks like modern data center racks,  
where TCP/IP and, more recently, RDMA~\cite{rocev2} have
become standard transport options. 
RDMA has been the preferred target transport 
in most of the recent systems because it cuts down on the 
software stack on both sides and, more so, because it 
provides one-sided accesses that avoid remote CPU in 
critical path. For example, systems that~\cite{literdma,aifm}
explored the TCP/IP option reported up to 10 $\mu$s overhead 
compared to RDMA for remote operations. Technologies 
like Intel Omnipath~\cite{omnipath} and CCIX~\cite{ccix}
are expected to further slash the latency by bringing the NIC 
much closer to the CPU, once they are commercially available. 
\anil{talk about the role of network congestion control?}

% (However, RDMA comes with its scalability constraints 
% when it comes to registering a lot of memory. 
% LITE optimized  minimized syscall overheads and 
% avoid memory copying)

% predictable latency transport for strict tail latency 
% requirements: net disagg evaluates a few transports and 
% better congestion control leads to reduced slowdown. 


\vspace{3pt}
\noindent \uline{Minimizing/avoiding software overheads.}
With respect to the remote access latency, the goal for the 
runtime is to get keep it as close as possible to the raw network 
latency and minimize any software overheads. The main software 
overhead comes from finding space for incoming remote data 
and deciding which data to evict to make that space 
(i.e., the typical cache miss handling), before returning to 
the application. 

Paging-based systems started with conventional disk-paging 
subsystem and over time improved it to target remote memory
where paging is more frequent and operates in microsecond
timescales instead of milliseconds~\cite{Lim2012}. A common 
optimization is to decouple allocation and eviction by 
maintaining a free list for newly allocated or fetched-in 
data and moving eviction to the background and off the 
critical path~\cite{Lim2012,leap}. 
For example, Fastswap~\cite{fastswap} offloads  
memory reclamation to a dedicated CPU core so that the 
application CPU can return to user space and continue its 
execution. Another (defacto) optimization is to allocate local 
pages for newly allocated memory and keep it local until evicted
(i.e., delayed write-back).
When bringing additional data along with the required data
(either because of prefetching or high data granularity to 
exploit locality), it is recommended to fetch the additional 
data separately outside the critical path~\cite{fastswap} to 
avoid head-of-the-line blocking.

Fundamentally though, as long as cache misses (page faults) are 
handled in software (the only alternative with traditional hardware; 
although LegoOS, with hardware modifications, still takes the
software route due to miss path complexity), the software miss path 
will inevitably cause context-switching, CPU cache pollution,
etc. that adds to the latency. 
Kona~\cite{kona}, for this reason, proposes new hardware 
primitives to offload this path to avoid above issues 
and also benefit from hardware speedup. 
The flip side, of course, is the
complexity of implementing part of the runtime and networking 
stack in hardware.\todo{refine last two sentences}. 

\vspace{3pt}
\noindent \uline{Hiding the latency.}
From a throughput (CPU utilization) standpoint, latency 
can be hidden by switching out the current thread and 
running a different thread while the remote data is fetched.
Kernel thread context switches are on the order of 
microseconds and may not result in much savings 
but user space runtimes with light-weight 
threads can use this approach, like AIFM did.~\cite{aifm}.
% Only works for compute-intensive applications.
% polling wastes CPU but context-switching hurts 
% the latencies and are too high as well.

% \vspace{3pt}
% \noindent \uline{Server-side latency}
% Depending on whether server-side CPU and software stack is 
% involved, it may add some latency to the datapath. Why would 
% such 

\subsubsection{Reducing remote accesses with RPCs.}
Some access patterns like traversals may be ill-suited for remote 
fetching and cause too many cache misses no matter how good 
caching/prefetching is, only good thing might be executing those 
operations closer to the memory using RPCs. 
AIFM gets most of its performance\todo{how much? try to quantify it.}
by sending such memory intensive operations of its data structures to 
the memory server. Semeru~\cite{semeru} does the same for 
garbage collecting the disaggregated Java Heap. This, however,
feels like a slippery slope as it adds complexity to the memory 
server that we don't want to, at least in the hardware-disaggregated 
setup where there is limited compute on memory nodes. 
Even in the traditional setup, we are still looking to decouple 
CPU from memory and a compute complexity on the memory side does 
not help with the disaggregation. 
\todo{needs work}

% \subsubsection{Effect of Sharing}
% Why DSMs fare worse. 
% \anil{Comment on/compare \& contrast the performance of 
% some existing systems. Effect of transparency? 
% AIFM vs Fastswap}