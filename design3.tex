
\subsection{Memory Management}
Similar to virtual memory interface on traditional servers, a 
system for system for disaggregated memory needs to pool the 
memory on multiple servers/nodes and present a unified interface 
for allocating and accessing this memory by hiding away the 
underlying physical locations. It should be further responsible 
for tracking available memory across the nodes, map it to an  
application on allocation and fetch the data when required. 
Most systems~\cite{legoos,remregions,kona} employ a 
global memory manager that maintains this metadata and works 
with agents/daemons on the (memory) nodes to find space for 
new allocations. Compute nodes query the manager for new allocations 
and cache the mappings to directly go to the relevant memory node 
during access time. Under the hood, the system should aim for better 
memory efficiency while providing good data path performance and 
scalability. Below, we discuss few factors that affect these 
goals.


\subsubsection{Memory backing or complementing}
In a hardware disaggregated setting, all application memory is 
allocated in the disaggregated memory and local DRAM is very small
and only used as cache i.e, all the app memory is backed by 
disaggregated memory. In a software-disaggregated setup, 
a similar approach can be taken by reserving a small amount 
of local DRAM on each node for local workspace and the rest is 
pooled from which memory is transparently allocated to all the 
applications. However, this fails to exploit local affinity in such 
a non-uniform memory access setting so all systems proposed for 
traditional setting prioritize 
local allocations and only expose unutilized memory for shared 
cluster use. Such disaggregated pool can then be used for 
complementing (not backing) the local memory either by transparently 
expanding local address space (i.e., remote paging systems) or 
through other interfaces like mmapped files~\cite{remregions}. 


\subsubsection{Memory tracking granularity.}
Just like paged memory, memory allocation and mapping is done 
is fixed-size units (slabs). Although extending page size (4KB) 
to remote memory would be a natural choice, 
it adds a significant 0.2\% space overhead 
for maintaining mapping metadata (e.g., 2 GB for a 1 TB 
region\cite{remregions}), so most systems end up using a much 
bigger slab size (128B~\cite{remregions} segments or ``multiple" 
pages~\cite{infiniswap,kona}). Local memory allocators can then 
manage these slabs for fine-grained allocations. 
However, bigger sizes may cause internal fragmentation or 
require contiguous segments at memory servers and hurt memory 
efficiency. None of the works, however, evaluate this aspect 
and slab sizes were arbitrary.


\subsubsection{Balancing memory usage across nodes.}
Remote memory used by an application is better if uniformly 
distributed across multiple memory nodes both to balance the 
access load and to minimize the performance impact in case of 
remote node failures. With a global memory manager, we can maintain 
the available memory across the memory nodes and properly direct 
memory allocation requests to both distribute memory footprint 
of application as well as balance overall memory usage across 
the memory nodes~\cite{legoos,remregions,kona}. However, both 
the overhead of constantly communicating with memory nodes 
and the centralized manager may present a scalability bottleneck.
Infiniswap~\cite{infiniswap} takes a decentralized approach 
where nodes requesting memory choose randomly from the list of 
available nodes. To help reduce imbalance, it opts for power of 
two choices~\cite{10.5555/924815} where instead of randomly 
picking one, two nodes are picked randomly and the one with the 
most available memory is chosen. However, it is unclear if this 
approach sustains the balance with increasing scale or over time.

% too speculative? not too big of a challenge for systems 
% to implement this?
% 
% \subsubsection{Static or Dynamic mappings}
% why is this needed?
% virtual remote addresses is important?
% having everything at central coordinator may add latency 
% to the critical path and affects scalability. but 
% caching may give outdated information.
% one-sided remote accesses help with latency/performance 
% but may affect the memory server's ability to flexibly 
% perform local memory management? 

% affects reclamation is s/w-disaggregated systems
% During memory access, the node contacts the cached local map 
% and directly goes to the relevant memory node, and avoid the 
% overhead of going to the (global) manager. As mappings 
% are generally static, nodes can use one-sided RDMA memory
% accesses to shorten the access time. 


\subsubsection{Memory reclamation on the server.}
As mentioned before, software-disaggregated systems need to 
balance available memory on each node between local and 
global memory (with priority for local use), and hence should 
should be amenable to reclamation to make space for expanding 
local usage. Infiniswap~\cite{infiniswap}, which 
uses remote memory as swap space, writes swapped out pages 
to both remote memory and disk, and hence can afford to drop the 
remote data during reclamation on a remote node. Reclamation is 
less of 
an issue in the hardware disaggregated setting but they may still 
need to swap out some memory to disk under extreme cluster 
memory pressure. An interesting, rather unexplored, question then
is which memory should be picked to drop/evict to minimize 
performance degradation. Traditional systems use LRU lists based 
on page access
information to swap out colder pages however such hotness information 
is either not available or, at best, distributed across 
compute nodes in a disaggregated system. 
Infiniswap~\cite{infiniswap} again uses power of choices to make an
informed guess (where the memory node queries a random subset of 
compute nodes for this hotness information and drop relatively 
colder pages) rather than randomly dropping any arbitrary pages.

% \subsubsection{Server-side complexity}
% how memory is managed on each memory server; some systems 
% completely ignore this part
% how does deallocation work?
% naming
% function shipping: e.g., AIFM or Semeru
% legoos moves mmu and tlb to memory nodes, how about other systems?

\subsubsection{Summary}
Memory efficiency in general is not very well evaluated in any of 
the systems so far (only Infiniswap~\cite{infiniswap} had a chart 
showing cluster memory utilization over time), perhaps 
because of the focus on application 
performance which is the prime roadblock to feasible disaggregation.
Userspace runtimes completely punt on the memory management aspect 
and assume that required memory is pre-allocated~\cite{semeru} or 
limit themselves to working with a single memory server~\cite{aifm}. 
While they spotlighted the performance advantages of app integration, 
these systems most certainly need further work 
on the memory management side to be considered feasible for adoption. 
% Would compressing colder pages be worth the memory efficiency?