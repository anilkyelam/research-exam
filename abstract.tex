
\begin{abstract}
Memory disaggregation improves memory utilization 
by decoupling CPU and memory allocations for 
applications while freeing memory-intensive 
applications from constraints imposed by memory limits 
on traditional servers.
As the network speeds in the local networks like 
modern datacenters inch closer to the DRAM speed,
there has been a recent proliferation of work in 
this space ranging from software solutions for 
pooling memory of traditional servers and providing it 
to applications to diaggregating the memory in hardware. 

In this report, we look at various recent approaches 
to memory disaggregation and study the   
important factors that guide their design. 
while referring to all of the recent systems. 
Such factors include programming model exposed to app, 
the runtime design to keep the app performance closer to 
local setting, and memory management to maximize 
utilization. 
We end with a discussion on the remaining 
challenges and future opportunities that 
can help make disaggregation more amenable 
for adoption. 

\end{abstract}