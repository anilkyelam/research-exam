
\begin{abstract}
Memory disaggregation addresses memory imbalance in a 
cluster by decoupling CPU and memory allocations of 
applications while also increasing the effective memory 
capacity for (memory-intensive) applications beyond the 
local memory limit imposed by traditional fixed-capacity 
servers. As the network speeds in the tightly-knit 
environments like modern datacenters inch closer to the 
DRAM speeds, there has been a recent proliferation of 
work in this space ranging from software solutions that 
pool memory of traditional servers for the shared use of 
the cluster to systems targeting the memory disaggregation
in the hardware. 
In this report, we look at some of these recent memory 
disaggregation systems and study the important factors 
that guide their design, such as the interface through 
which the memory is exposed to the application, the 
runtime implementation and relevant optimizations to 
retain the near-native application performance, 
various approaches to the cluster memory management to 
maximize memory utilization, etc. and analyze the 
associated trade-offs. We conclude with a discussion 
on some open questions and potential future directions  
that can render disaggregation more amenable for adoption.

\end{abstract}