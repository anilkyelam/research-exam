
\begin{abstract}
Memory disaggregation addresses memory imbalance  
in a cluster by decoupling CPU and memory allocations 
of applications while increasing the effective 
memory capacity for memory-intensive 
applications beyond the local memory limit  
imposed by traditional fixed-capacity servers.
As the network speeds in the local networks like 
modern datacenters inch closer to the DRAM speed,
there has been a recent proliferation of work in 
this space, ranging from software solutions that  
pool memory of traditional servers and make it 
available to the cluster, to systems targeting 
the memory disaggregated in hardware. 
In this report, we look at some of these approaches 
towards memory disaggregation systems and study in 
detail the important factors that guide their design;
such as
the interface through which the memory is exposed 
to the application, the implementation of the runtime 
and relevant performance optimizations to retain 
the near-native application performance, 
the various approaches 
to the global memory management with a goal of 
maximizing memory utilization, etc. and study the 
associated trade-offs.
We conclude with a discussion on the further  
challenges and future opportunities that 
can help make disaggregation more amenable 
for adoption. 

\end{abstract}